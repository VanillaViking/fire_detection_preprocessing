\documentclass[lettersize,journal]{IEEEtran}
\usepackage{amsmath,amsfonts}
\usepackage{algorithmic}
\usepackage{algorithm}
\usepackage{array}
\usepackage[caption=false,font=normalsize,labelfont=sf,textfont=sf]{subfig}
\usepackage{textcomp}
\usepackage{stfloats}
\usepackage{url}
\usepackage{verbatim}
\usepackage{graphicx}
\usepackage{cite}
\hyphenation{op-tical net-works semi-conduc-tor IEEE-Xplore}
% updated with editorial comments 8/9/2021

\begin{document}

\title{Significance of Image Pre-processing in Computer Vision Based Fire Detection, A Review}

\author{Ashwin Rajesh,
        % <-this % stops a space
\thanks{This paper was produced by the IEEE Publication Technology Group. They are in Piscataway, NJ.}% <-this % stops a space
\thanks{Manuscript received April 19, 2021; revised August 16, 2021.}}

% The paper headers
\markboth{Journal of \LaTeX\ Class Files,~Vol.~14, No.~8, August~2021}%
{Shell \MakeLowercase{\textit{et al.}}: A Sample Article Using IEEEtran.cls for IEEE Journals}

\IEEEpubid{0000--0000/00\$00.00~\copyright~2021 IEEE}
% Remember, if you use this you must call \IEEEpubidadjcol in the second
% column for its text to clear the IEEEpubid mark.

\maketitle

\begin{abstract}
This document describes the most common article elements and how to use the IEEEtran class with \LaTeX \ to produce files that are suitable for submission to the IEEE.  IEEEtran can produce conference, journal, and technical note (correspondence) papers with a suitable choice of class options. 
\end{abstract}

% \begin{IEEEkeywords}
% Article submission, IEEE, IEEEtran, journal, \LaTeX, paper, template, typesetting.
% \end{IEEEkeywords}

\section{Introduction}
\IEEEPARstart{W}{ildfires} pose a significant threat to humans, wildlife and the
environment alike. Left unchecked, A wildfire can spread rapidly causing
large scale destruction to forests \& infrastructure, as well as
releasing large amounts of pollutants into the atmosphere. Therefore, it
has become increasingly crucial to detect them as early as possible.
This report aims to explore different methods employed in improving the
performance and efficiency of wildfire detection systems, with a focus
on computer vision based fire detection using Convolutional Neural
Networks (CNNs). Fire detection is carried out through a variety of
mediums, such as terrestrial sensor nodes, UAV scanning and satellite
based approaches. Observing the literature, it is apparent that computer
vision plays a significant role in all systems, Making it important to
maximise their efficiency and accuracy. Furthermore, it is important to
optimise such systems to run on low-powered edge computing devices,
which are currently used to detect wildfires. To this end, a multitude
of image preprocessing filters relevant to fire detection are explored
report.

This paper contributes a benchmark of different pre-processing filters
and algorithms to uncover insights on an ideal pipeline for wildfire
detection, using metrics such as speed, power consumption and model
accuracy. The benchmarks are carried out on a Raspberry Pi 4B, to
similate low-powered edge computing hardware that is consistently used
in terrestrial, UAV and satellite systems. Large deep-learning networks
are unviable on these systems due to the high computational intensity,
while models with reduced parameters are more efficient but suffer from
less accurate inference. The preprocessing pipeline aims to improve the
accuracy of lightweight models by highlighting important features in
fire and smoke, while reducing unwanted noise in the image.

Furthermore, an improved system for early smoke detection is
investigated. The proposed dark channel prior + edge detection algorithm
aims to improve on existing methods of smoke preprocessing using DCP
detailed in \cite{prepfire}. Specifically, false positives due to high
light intensity artifacts in the image such as the sky are significantly
mitigated, thanks to edge detection filters revealing characteristics
that are unique to smoke \cite{wsnfire}.

\bibliographystyle{IEEEtran}
\bibliography{ref}

\newpage


\vfill

\end{document}


